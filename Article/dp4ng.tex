Differentiable Programming is a paradigm in which automatic differentiation is applied to general computer programs. It began after the popularity of structured or physics-informed neural networks led to increasing interest in using dynamical systems theory in machine learning e.g., (\cite{chen2019}).

Differentiable Simulation (DS) is a particular case, where the program of interest is a dynamical systems model. It can be thought as one implementation of PIML, wherein the data (usually spatiotemporal) is thought to be described by a partially known ODE. In applications where the underlying physics is partially known, DS can be \textit{very} efficient, and like other applications of PIML, can improve generality by removing out-of-distribution sampling issues present in black-box NNs. Further, learning is likely more rigorous, as the known-portion of physics is as well approximated as in a direct simulation.

Our example of DS is in enriching unit commitment with knowledge of transient states of the gas system model. To do this, we need to illucidate both aspects.

\subsection{Gas Dynamics}

\subsection{Optimal Power Flow}
