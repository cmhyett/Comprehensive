We utilize dynamic modeling discussed in \cite{osiadacz_simulation_1984,steinbach_pde_2007,Chertkov2015Cascading} (see also references therein)  and specific algorithmic scheme from \cite{Gyrya2019AnExplicit} called \href{https://github.com/kaarthiksundar/GasTranSim.jl}{GasTranSim.jl} implemented by Los Alamos National Laboratory team in Julia \cite{GasTranSim}. 

Basic equations, expressing conservation of mass and momentum and  describing flow of gas in a single pipe (with gas velocity much smaller than the speed of sound), are  \cite{osiadacz_simulation_1984,steinbach_pde_2007}:
\begin{align}\label{eq:mass}
    & \partial_t \rho + \partial_x \phi = 0,\\ \label{eq:momentum}
    & \partial_t \phi + \partial_x p = -\frac{\lambda}{2D} \frac{\phi |\phi|}{\rho},
\end{align}
% \begin{align}\label{eq:mass} & \partial_t \rho + \partial_x(\rho u) = 0,\\ \label{eq:momentum}
% & \partial_t (\rho u) + \partial_x (\rho u^2) = -\partial_x p - \lambda \rho u |u|,
% \end{align}
where $\rho(t;x)$, $\phi(t;x)$ and $p(t;x)$ are the gas density, mass flux and pressure measured at the moment of time $t$ at the position $x$ along the pipe; $\lambda$ is the Darcy-Weisbach friction factor of the pipe (per diameter, $D$).
Eqs.~(\ref{eq:mass}) -(\ref{eq:momentum}) are supplemented by the equation of state, relating pressure and density
\begin{equation}
    p = Z(\rho,T)RT\rho
\end{equation}
In this study we use the CGNA formula for $Z(\rho, T)$, see e.g. \cite{menon}.
\begin{comment}
\begin{gather}
\label{eq:state} p = Z RT\rho,
\end{gather}
where the compressibility factor, $Z$, dependent on temperature  is a function of temperature computed according to the standard CGNA formula \cite{menon}.
%\misha{It may be better to add equation of state, comment on the choice of parameters, and cite a paper. CGNA?}.
%We calculate the compressibility factor $Z(\rho, T)$ with the CNGA method\cite{menon}, which expresses as follows 
\end{comment} 
Effects of gravity and temperature variations along the system are ignored, as the effects are small and our representation is schematic
\footnote{Taking into account the temperature variations is only important in countries with sub-zero winter temperatures. % like Canada or Russia. 
In Israel, this is not the case and we make the assumption of a constant temperature equal to 15 Celsius degrees. Accounting for the effects of elevation/gravity is of a concern only at the node \#6, located 
%Concerning the lack of gravity effects, the sole network point which could be problematic in the system is node 6, which is situated 
in the Dead Sea area, roughly 400 meters below the sea level, and at the node \# 5, located close to  Beer-Sheva at 300 meters above the sea level. Given that ignorning this effect results only in a relatively minor pressure drop of $3$ to $5$ bars in the southern part of the system, we ignore it for now in the simplest version of the model.}.

The single pipe description extends to a system of pipes. Each pipe is characterized by three parameters: diameter, length, and the friction factor per diameter. Each node prescribes a boundary condition for one side of (at-least) one pipe at all instances in time.
Additionally, nodes and pipes are joined via condition of mass conservation (Kirchoff's rule, that the mass entering a junction must equal the mass exiting the junction).
As is standard, demand nodes specify a flux withdrawal as a function of time. The system in question however, additionally specifies flux at supply nodes. Thus all supplied boundary conditions are on flux.

Denoting $\rho_{ij}, \phi_{ij}$ to be the dynamic variables on the pipe from node $i$ to node $j$, and $\rho_n,\phi_n$ to be the density and flux at node $n$, we write the full system to be solved as:
\begin{align}
    \partial_t \rho_{ij} + \partial_x \phi_{ij} &= 0\\
    \partial_t \phi_{ij} + \partial_x p_{ij} &= -\frac{\lambda_{ij}}{2D}\frac{\phi_{ij}|\phi_{ij}|}{\rho_{ij}}\\
    \text{subject to initial}& \text{ and boundary conditions:} \nonumber\\
    \rho_{ij}(x,0) &= \rho_{0,ij}(x)\\
    \phi_{ij}(x,0) &= \phi_{0,ij}(x)\\
    \phi_n(t) &= d_n(t)\\
    \sum_{j \in \mathcal{E}} \phi_j S_{ij} + d_{j} &= 0
\end{align}

Where $S_{ij}$ is the cross-section of the pipe. Initial conditions for density and mass-flux in the system are constructed based on actual operational data. To solve for dynamics of mass flows and pressures across the system we use the staggered-grid approach of \cite{Gyrya2019AnExplicit} which is an explicit, conservative, second order, finite difference scheme, stable given a CFL condition is satisfied.
We remind that, as of now, the  Israel system does not contain compressors.
\begin{comment}
This approach solves the governing system of PDEs, \ref{eq:governingEq1},\ref{eq:governingEq2} on two separate grids.

Then, on these grids we can write \(\rho(\tau_n,\xi_j) = \rho_n^j\). Using finite difference, and defining \(\beta = \frac{\lambda L}{2D}\), we write the dynamical equations \ref{eq:governingEq1}, \ref{eq:governingEq2} in the discretized forms:
\begin{align}
  &\frac{\rho^{n+1}_j - \rho^n_j}{\Delta \tau} + \frac{\phi^m_{j} - \phi^m_{j-1}}{\Delta \xi} = 0\\
  &\frac{\phi_j^{m+1} - \phi_j^m}{\Delta \tau} + \frac{\rho_{i+1}^{n+1} - \rho_i^{n+1}}{\Delta \xi} = -\beta \frac{ (\phi |\phi |)^{n+1}_j}{\rho_j^{n+1}} \approx -\beta \frac{ (\phi |\phi |)^{m+1}_j + (\phi |\phi |)^{m}_j}{\rho_{i+1}^{n+1} + \rho_{i}^{n+1}} \label{eq:discretizedMomentum}\\
  & \qquad \implies \phi_j^{m+1} + \beta \Delta \tau \frac{ (\phi |\phi|)^{m+1}_j}{\rho_i^{n+1} + \rho_{i+1}^{n+1}} =
    \phi_j^m - \frac{\Delta \tau}{\Delta \xi}\left( \rho^{n+1}_{i+1} - \rho_i^{n+1} \right) - \beta \Delta t \frac{ (\phi |\phi|)^{m}_j}{\rho_i^{n+1} + \rho_{i+1}^{n+1}}
\end{align}


We can solve the former to find an equation to advance \(\rho\):
\begin{equation}
  \rho^{n+1}_j = \left( \frac{\phi^m_{j-1} - \phi^m_{j}}{\Delta \xi} \right) \Delta \tau + \rho^n_j\\
\end{equation}

However, the discretized momentum equation \ref{eq:discretizedMomentum} is implicit.

Labeling \(y\) as the RHS of \ref{eq:discretizedMomentum}, i.e.,
\begin{equation}
y = \phi_j^m - \frac{\Delta \tau}{\Delta \xi}(\rho_{i+1}^{n+1} - \rho_i^{n+1}) - \beta \Delta \tau \frac{ (\phi |\phi|)^m_j}{\rho_i^{n+1} + \rho_{i+1}^{n+1}}
\end{equation}
We have an equation of the form:
\begin{align}
  &F(x) = y \\
  &x = \phi_j^{m+1} \\
  F(x) = x + \beta \Delta \tau \frac{ (x |x|)}{\rho_i^{n+1} + \rho_{i+1}^{n+1}}
\end{align}

We can write the inverse of \(F\) as
\begin{equation}
x = F^{-1}(y) = \text{sign}(y)\frac{-1 + \sqrt{1+4a|y|}}{2a}
\end{equation}

And because we know the value of \(y\), we can readily solve for \(x := \phi_j^{m+1}\).

\end{comment}


