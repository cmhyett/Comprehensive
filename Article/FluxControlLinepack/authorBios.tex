
\textbf{Criston Hyett} - is a Ph.D. student in Applied Mathematics at the University of Arizona. He is advised by Misha Chertkov, and interested in data-enhanced dynamical systems modeling, encompassing physics-informed machine learning, reduced order modeling and uncertainty quantification.

\noindent
\textbf{Laurent Pagnier} - is a Visiting Assistant Professor at the University of Arizona. His main research interest is the application of Machine Learning techniques to the operation of large infrastructures. He is particularly interested in reinforcing the interpretability and trustworthiness of ML methods which are paramount to increase their acceptance and usage by practitioners.

\noindent
\textbf{Jean Alisse} - received his Phd in Physics in 1999 (Paris University). He worked during 11 years at the Israel Electric Company in the Planning and Division company.
There he headed a Modeling group which focused on CFD problems and Natural Gas dynamics. Since the creation of Noga, the Israel Independent System Operator, he has been pursuing the same works, with a stress on developing models for Gas dynamics simulation and Gas-Power systems. 

 \noindent
\textbf{Lilach Sabban} - received her PhD from the Technion Israel Institute of Technology. Lilach specializes in fluid dynamics as a mechanical engineer. She is a researcher at Noga, Israel Independent System Operator,  involved in a range of renewable energy projects and has been investigating natural gas dynamics.

\noindent
\textbf{Igal Goldshtein} - received his B.Sc from the Technion Israel Institute of Technology. He worked at Israel electric company for 12 years, seven of them as Gas-Turbine dispatcher in the system operator control center. During the last two years, at Noga, the Israel System Independent Operator, he handles the Operator Training System (OTS). He also researches operation of Israel electric system under various stress conditions.

\noindent
\textbf{Michael (Misha) Chertkov} - is Professor of Mathematics and chair of the Graduate Interdisciplinary Program in Applied Mathematics at the University of Arizona since 2019. He focuses in his research on foundational problems in mathematics and statistics applied to physical systems, in particular fluid mechanics, to engineered systems such as energy grids, and to some bio-social systems. Dr. Chertkov received his Ph.D. in physics from the Weizmann Institute of Science in 1996, spent three years at Princeton University as a R.H. Dicke Fellow in the Department of Physics, and joined Los Alamos National Laboratory in 1999, initially as a J.R. Oppenheimer Fellow and then as a Technical Staff Member in Theory Division. He has published more than 250 papers, is a fellow of the AAAS, a fellow of the American Physical Society and a senior member of IEEE.
