We investigate the spatiotemporal response of a reduced model of Israel's NG network to prescribed insults and human-in-the-loop controls in order to evaluate robustness and suggest control strategies. To reiterate, Israel's network is unique because of the absence of a compressor, and that the inlets specify flux, not pressure. Further, we perform this study looking towards the increased importance of NG to mitigate increasing stochasticity in demands expected in the coming years as coal is phased out, and renewables grow.

The specification of flux vs pressure leads to the pressure timeseries of the network being dominated by daily demand curves as shown in Fig.~(\ref{fig:scen1}), increasingly susceptible to pressure drift from stochastic fluctuations in nominal demands.

Further, we call out the importance of robustness of the network not simply to insults, but to insults at any time - leading to the idea of "system reserve" being time and spatially dependent.

Future work will improve on modeling to more completely capture uncertainty propagation through the network, and its influence and interaction with control strategies. We envision extending the prescribed control, also reinforced by monotonicity \cite{Vuffray2015Monotonicity,zlotnik_monotonicity_2016}, developed in this manuscript with the powerful optimization approaches developed to account for dynamic optimization over compressors \cite{rachford_optimizing_2000,carter_optimizing_2003,rachford_using_2009,zlotnik_optimal_2015,zlotnik_using_2016}, e.g. to evaluate benefits of adding compressors to the NG system of Israel. We also plan to carry on a comprehensive modeling and control of the combined power and gas system of Israel, in the spirit of the approach highlighted in \cite{carter_impact_2016,Zlotnik2017Coordinated}. 